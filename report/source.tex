% !TEX program = xelatex

\documentclass[hidelinks, 12pt, a4paper]{article}

\usepackage{fontspec}
\setmainfont[Ligatures=TeX]{Linux Libertine O}

\usepackage[hidelinks, colorlinks = true, urlcolor = blue]{hyperref}

\usepackage[utf8]{inputenc}
\usepackage{indentfirst}
\usepackage{graphicx}
\usepackage[left=2cm,right=2cm,top=2cm,bottom=2cm]{geometry}
\usepackage{lipsum}
\usepackage{caption}
\usepackage{subcaption}
\usepackage{verbatim}


\begin{document}
\sloppy % this is legendary

\input{titlepage}
%\maketitle


\pagebreak
\tableofcontents
\pagebreak


\section{Δομή του προγράμματος}

\vspace{1cm}

\begin{verbatim}
├── applications
│   ├── Copter.java
│   ├── Echo.java
│   ├── Media.java
│   └── Obd.java
├── plots
│   └── plot.py
├── stamps
│   ├── audio.txt
│   ├── auto.txt
│   ├── copter_tcp.txt
│   ├── copter.txt
│   ├── echo.txt
│   ├── https.txt
│   ├── image.txt
│   ├── obd_tcp.txt
│   ├── obd.txt
│   ├── temp.txt
│   ├── test.txt
│   └── welcome.txt
└── UserApplication.java
\end{verbatim}


\begin{itemize}
    \item Το αρχείο που βρίσκεται η main είναι το \emph{UserApplication.java}.
    \item Στον φάκελο \emph{applications}, δημιουργήσαμε ξεχωριστά αρχεία για κάθε εφαρμογή.
    \item Στον φάκελο \emph{stamps}, βρίσκονται οι έξοδοι του προγράμματος figlet για ascii art λόγους, όπως έχει αναφερθεί και στο report!
    \item Στον φάκελο \emph{plots}, έχουμε τέλος ένα python αρχείο για να δημιουργήσουμε τα διαγράμματα μας.
\end{itemize}

Στη συνέχεια θα επικολλήσουμε μόνο τον κώδικα των \emph{applications} και του \emph{UserApplication.java}...

\pagebreak

\section{UserApplication.java}

\section{applications}

\subsection{Echo.java}
\subsection{Media.java}
\subsection{Obd.java}
\subsection{Copter.java}

\end{document}
