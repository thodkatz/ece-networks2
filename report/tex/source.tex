% !TEX program = xelatex

\documentclass[hidelinks, 12pt, a4paper]{article}

\usepackage{fontspec}
\setmainfont[Ligatures=TeX]{Linux Libertine O}

\usepackage[hidelinks, colorlinks = true, urlcolor = blue]{hyperref}

\usepackage[utf8]{inputenc}
\usepackage{indentfirst}
\usepackage{graphicx}
\usepackage[left=0.5cm,right=0.5cm,top=2cm,bottom=2cm]{geometry}
\usepackage{lipsum}
\usepackage{caption}
\usepackage{subcaption}
\usepackage{verbatim}


\usepackage{listings}
\usepackage{xcolor}

\definecolor{codegreen}{rgb}{0,0.6,0}
\definecolor{codegray}{rgb}{0.5,0.5,0.5}
\definecolor{codepurple}{rgb}{0.58,0,0.82}
\definecolor{backcolour}{rgb}{0.95,0.95,0.92}

\lstdefinestyle{mystyle}{
    backgroundcolor=\color{backcolour},   
    commentstyle=\color{codegreen},
    keywordstyle=\color{magenta},
    numberstyle=\tiny\color{codegray},
    stringstyle=\color{codepurple},
    basicstyle=\ttfamily\footnotesize,
    breakatwhitespace=false,         
    breaklines=true,                 
    captionpos=b,                    
    keepspaces=true,                 
    numbers=left,                    
    numbersep=5pt,                  
    showspaces=false,                
    showstringspaces=false,
    showtabs=false,                  
    tabsize=2
}

\lstset{style=mystyle}



\begin{document}
\sloppy % this is legendary

\input{titlepage}
%\maketitle


\pagebreak
\tableofcontents
\pagebreak


\section{Δομή του προγράμματος}

\vspace{1cm}

\begin{verbatim}
├── applications
│   ├── Copter.java
│   ├── Echo.java
│   ├── Media.java
│   └── Obd.java
├── plots
│   └── plot.ipynb
├── ascii
│   ├── audio.txt
│   ├── auto.txt
│   ├── copter_tcp.txt
│   ├── copter.txt
│   ├── echo.txt
│   ├── https.txt
│   ├── image.txt
│   ├── obd_tcp.txt
│   ├── obd.txt
│   ├── temp.txt
│   ├── test.txt
│   └── welcome.txt
└── UserApplication.java
\end{verbatim}


\begin{itemize}
    \item Το αρχείο που βρίσκεται η main είναι το \emph{UserApplication.java}.
    \item Στον φάκελο \emph{applications}, δημιουργήσαμε ξεχωριστά αρχεία για κάθε εφαρμογή.
    \item Στον φάκελο \emph{ascii}, βρίσκονται οι έξοδοι του προγράμματος figlet για ascii art λόγους, όπως έχει αναφερθεί και στο report!
    \item Στον φάκελο \emph{plots}, έχουμε τέλος ένα python αρχείο για να δημιουργήσουμε τα διαγράμματα μας.
\end{itemize}


\pagebreak

\section{UserApplication.java}

\lstinputlisting[language=Java]{../src/UserApplication.java}

\section{applications}

\subsection{Echo.java}

\lstinputlisting[language=Java]{../src/applications/Echo.java}

\subsection{Media.java}

\lstinputlisting[language=Java]{../src/applications/Media.java}

\subsection{Obd.java}

\lstinputlisting[language=Java]{../src/applications/Obd.java}

\subsection{Copter.java}

\lstinputlisting[language=Java]{../src/applications/Copter.java}

\end{document}
