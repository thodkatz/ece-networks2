% !TEX program = xelatex

\documentclass[hidelinks, 12pt, a4paper]{article}

\usepackage{fontspec}
\setmainfont[Ligatures=TeX]{Linux Libertine O}

\usepackage[hidelinks, colorlinks = true, urlcolor = blue]{hyperref}

\usepackage{indentfirst}
\usepackage{graphicx}
\usepackage[left=2cm,right=2cm,top=2.5cm,bottom=2.5cm]{geometry}
\usepackage{lipsum}

%\setlength{\parindent}{1em}
%\setlength{\parskip}{1em}\title{Εργασία Στατιστικής}

%\title{Java socket programming \\ \includegraphics[width=\textwidth]{assets/login.png}}
% \author{Θεόδωρος Κατζάλης \\ ΑΕΜ:9282 \\ katzalis@auth.gr}
% \date{19/07/2020}

\begin{document}

\input{titlepage}
%\maketitle


\pagebreak
\tableofcontents
\pagebreak

% \section{Lorem}
% \lipsum


\section{Εισαγωγή}

Η συγκεκριμένη εργασία αποσκοπεί στην εξοικείωση εννοιών σχετικά με τα δίκτυα υπολογιστών τόσο σε θεωρητικό όσο και σε πρακτικό επίπεδο. Το πρώτο εξασφαλίζεται με την μελέτη και παράθεση βιβλιογραφίας σε πρωτόκολλα επικοινωνίας \textbf{TCP} και \textbf{UDP} ενώ το δεύτερο εξασφαλίζεται, χρησιμοποιώντας την γλώσσα προγραματισμού \textbf{java}, με την δημιουργία δικτυακών εφαρμογών συλλογής πληροφοριών σε συνεργασία με τον server του μαθήματος ithaki με IP 155.207.18.208.

\section{Session 1}

\section{Section 2}

\section{UDP}
Hello

\section{Audio streaming protocols}

\section{Πηγαίος κώδικας}

\subsection{User Interface}

\begin{figure}[h!]
\centering
	\includegraphics[width=\textwidth]{assets/ui.png}
	\caption{Εκτέλεση προγράμματος σε κέλυφος Bash} 
    \label{fig:ui}
\end{figure}



\section{Statement of originality aka bibliography}
 
\end{document}
